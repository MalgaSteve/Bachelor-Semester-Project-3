\documentclass[../main.tex]{subfiles}
\usetikzlibrary{positioning, arrows.meta}

\begin{document}
Honeywords assumes the connection runs over servers using TLS, which has several
problems including that the trust of a third party is needed. It is also more
prone to phishing if the attacker succeeds in disguise as the server that
distributes certificates or by tricking the user. To solve this problem,
Arriaga, Ryan, and Skrobot \cite{marjan2023} propose the SweetPAKE protocol, which
combines the strength of a PAKE protocol and honeywords. First, they propose a
naive approach using the SPAKE protocol however it is not efficient. They then
propose a secure SweetPAKE protocol using PAPKE, they call it BeePAKE. 

Note: The authors of SweetPAKE highly recommend using MAC authentication during
the protocol but this report does not include MAC because it is out of the scope of
this project and is not implemented in the technical part.

\subsection{Setup} Let two parties be Alice and Bob, both share a secret key,
here password \(pwd\). Let \(Gen\),  \(Enc\), and \(Dec\), be three functions
of a secure PAPKE encryption scheme. Let \(PRF\) be a secure pseudo-random
function which returns a vector of randomly generated long-term keys.

\subsection{Process}
\subsubsection{} Alice generates \((apk, sk)\) using the \(Gen\) function of
PAPKE is explained in the section above.

\[(apk, sk) \leftarrow Gen(pwd)\]

She sends her \(id\) and \(apk\) to Bob.

\subsubsection{} Bob generates a random key \(k\) that is used to generate a 
vector of long-term keys \(K\).

\[k \leftarrow {0,1}^*\]
\[K = PRF(k, (id_a, apk))\]

Bob then creates a vector \(C\) by generating secrets \(c\) using \(Enc\) for 
every sugarword in the password file.

\begin{algorithmic}
  \For{\(i = 1, ..., n\)}
	\[C[i] \leftarrow Enc(pwd, apk, K[i])\]
  \EndFor
\end{algorithmic}

After that he does a random permutation \(RP\) of vector \(C\), meaning shuffles the
vector, such that the position of the correct password remains unknown even for
Alice. He then stores the mapping \(pmap\) of the permutation.

\[(C', pmap) = RP(C)\]

The vector \(C\) will then be sent to Alice together with Bobs' \(id\).

\subsubsection{} Alice uses the \(Dec\) PAPKE function for every element in
vector \(C\). If the function cannot decrypt a valid key the process is 
aborted. After a successful decryption, Bob stores the position \(i\). 

\begin{algorithmic}
  \For{\(i = 1, ..., n\)}
	\[k' \leftarrow Dec(sk, C'[i])\]
  \EndFor
  \If{\(k' = undefined\)}
    abort
  \EndIf
\end{algorithmic}

Alice computes the long-term session key and sends the index \(i\) to Bob. 

\[tr \leftarrow (id_a, id_b, apk, C, i)\]
\[key \leftarrow PRF(k', tr)\]

\subsubsection{} Bob then computes with help of the stored mapping \(pmap\) the
correct position of the password of the user and the key. 

\[tr \leftarrow (id_c, id_b, apk, C, i)\]
\[key \leftarrow PRF(K[i], tr)\]

After the key computation, the server sends the honeychecker (explained in section
Honeywords) the clients' user id and the position \(i\). The honeychecker then
checks if \(i\) is the position of the correct password, if it is not the
honeychecker alarms the system administrator.

\end{document}
