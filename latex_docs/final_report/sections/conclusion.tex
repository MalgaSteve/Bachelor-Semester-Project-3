\documentclass[../main.tex]{subfiles}

\begin{document}

In conclusion, the report delves into the challenges of traditional username
and password authentication methods and explores alternative approaches to
improve security. The report analyzes concepts of honeywords proposed by Rivest
and Juels \cite{juels2013honeywords}, which incorporates decoy passwords in the
password file to detect data breaches. It also discusses Password Authenticated
Key Exchange protocols, specifically focusing on the PAPKE protocol proposed by
Bradley, Tatiana and Camenisch, Jan and Jarecki, Stanislaw and Lehmann, Anja
and Neven, Gregory and Xu, Jiayu \cite{bradley2019password}.

The SweetPAKE protocol, proposed by Arriage, Ryan, and Skrobot, combines the
strengths of both worlds PAKE protocols and honeywords to address the
challenges faced in authentication. The authors present BeePAKE as a secure
variant of SweetPAPKE using the PAPKE protocol.

How to detect if a password file is in possession of intruders and at the same
Does time prevent phishing attacks?

Combining the strength of honeywords and PAKE, the SweetPAKE protocol is a
promising answer. By implementing a secure variant of SweetPAKE, here BeePAKE,
the report demonstrates a practical application that enhances security in
password-based authentication. The protocol provides a way to detect intruders
through honeywords and also ensures secure communication using long-term keys
generated from weak passwords using PAKE protocols. By using a PAKE protocol,
one must not trust a third party being a server that distributes certificates,
as a consequence, an attacker can't masquerade as such thus preventing phishing
attacks.

\end{document}
