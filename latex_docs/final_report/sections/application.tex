\documentclass[../main.tex]{subfiles}

\begin{document}

This section covers the technical part of this project. It consists of an
implementation of a BeePAKE protocol \cite{marjan2023}. 

\subsection{Requirements}
The application should be an implementation of BeePAKE \cite{marjan2023} which
has been discussed and analyzed in the previous section. The implementation
should work on every operating system with Python version 2 or higher
installed. The implementation will implement BeePAKE with the PAPKE-FO
\cite{bradley2019password} protocol explained in previous sections. The project
will include an example file that will show how the implementation can be used
and tested. It will have two Parties and demonstrate how both share a key using
the code provided.

\subsubsection{Performance}
Performance is important since the goal of BeePAKE is to provide people and 
enterprises with the ability to have a secure-by-design protocol without being
limited by poor performance. The process of sharing the key should not take more 
than \(25\%\) time in seconds of \(k\) sugarwords. Let \(T\) be the acceptable time
requirement. 

\[T = 25\% \cdot k\]

As an example, a client and a server having 20 sugarwords should not take more than
5 seconds to share the session key.

\subsubsection{Functional requirements} 
The protocol will be split into four parts each representing one step of the
protocol. Each part has its own function:

\begin{itemize}
	\item generate()
	\item encryption()
	\item decryption()
	\item retrieve\_key()
\end{itemize}

The protocol also needs three functions of the encryption scheme of the PAPKE protocol:

\begin{itemize}
	\item papke\_generate()
	\item papke\_encryption()
	\item papke\_decryption()
\end{itemize}

The function generate() should use the following template:
\begin{itemize}
	\item Description: First step of the BeePAKE protocol
	\item Parameter: No parameters
	\item Pre-condition: Protocol was started
	\item Post-condition: Generates a secret key, and a public key and returns the outbound message which will be sent
	\item Trigger: A party requests key-sharing with a server
\end{itemize}

\subsection{Design} 
The implementation uses Python because it is easily readable and has a lot of great
cryptographic libraries. As a basis for the project, modules of the GitHub
repository python-spake2 created by user warner \cite{warner2016} were used. The
modules used are groups.py, six.py, util.py and spake.py, it is an
implementation of the SPAKE2 protocol. 

The implementation splits the BeePAKE protocol into four steps and two parties
\(A\) and \(B\).

First step: The first party \(A\) generates a secret key and public key using its'
password. It then stores the secret key and sends the first message to the
other party \(B\). The message includes the id of the party \(A\) sending it and the public
key.

Second step: The second party \(B\) receives the messages, and parses them to get the
public key and the id of the other party \(A\). Then, he begins to generate the
secrets, by encrypting a randomly generated session key for every sugarword using
the encryption process of the PAPKE protocol. Party \(B\), then sends his id
and the generated secrets.

Third step: The first party \(A\) parses the received message and decrypts
every key until a valid key is found. The resulting key is then stored and the party
\(A\) sends the index of the valid key of the set of received keys including
his id to party \(B\). 

The last step: Party \(B\) parses the received message, retrieves the index and
asks the honeychecker to check if it is a honeyword. He then stores the associated
session key.

\subsubsection{File structure} 
The file structure is an important part of the project since modules refer to other files and use
other modules within the project. The main folder has two main directories, one example Python
file and one example password file.

\begin{itemize}
	\item honeyword\_generation (directory)
	\begin{itemize}
		\item gen.py (python file)
		\item c\_pws (text file)
	\end{itemize}

	\item sweet\_pake (directory)
	\begin{itemize}
		\item file\_operation.py (python file)
		\item groups.py (python file)
		\item \_\_init\_\_.py (python file)
		\item six.py (python file)
		\item sweet\_pake.py (python file)
		\item util.py (python file)
	\end{itemize}
	\item client\_sweetPake.py (python file)
	\item pw\_file (text file)
\end{itemize}

The project is split into two main directories "honeyword\_generation" and
"sweet\_pake". The "honeyword\_generation" folder includes a text file "c\_pws"
having a fairly big list of the most common passwords and a Python file "gen.py"
whose main purpose is to append or change a users' entry in a password file.
The Python file has multiple honeyword generation algorithms implemented that
were proposed by Rivest and Juels \cite{juels2013honeywords}. Some of these
algorithms use the mentioned text file "c\_pws". The directory
"honeyword\_generation" is meant to be used to create an example password file
to test the implementation of BeePake.

The directory "sweet\_pake" is the heart of the project, the implementation of
BeePAKE. It consists of 6 files, beginning with the file six.py which is a
library written by Benjamin Peterson \cite{peterson2010}. The library makes it
possible to make code of Python version 3, compatible with Python version 2
without having to declutter code. 

The "\_\_init\_\_.py" file is used as a package initializer, which means
that the current directory "sweet\_pake" is treated as a Python package. Packages
are treated like modules but the difference is that a module is one single file
and packages can contain several modules.

The "util.py" file is a Python file which consists of various utility function
which are needed for other modules in the directory. 

The "file\_operation" module contains functions which are related to writing or
reading file operations. 

The "groups.py" file is the implementation of the math behind the protocol
meaning the implementation of cyclic groups and various hashing algorithms. The major
part of this module was created by user warner of Git Hub \cite{Warner2013}.

Finally, the "sweet\_pake.py" module consists of the actual BeePAKE protocol.
It implements the various functions and logic of the protocol using the modules
mentioned above including libraries os, hashlib and hkdf from the standard library.

\subsubsection{Classes}
The different data classes:

sweet\_pake.py: The module is mainly split into two classes and various
Exception classes. Beginning with by presenting the error classes.
\begin{itemize} 
	\item SweetPAKEError(Exception) is used as an abstract class every
			SweetPAKE exception will derive from this one 
	\item OffSides(SweetPAKEERROR) is triggered if the message comes from
		the same side as received
	\item WongGroupError(SweetPAKEERROR) is triggered when a different or
		the invalid group is used
	\item ReflectionThwarted(SweetPAKEERROR) is triggered when someone tries to reflect the message back to the user
	\item OnlyCallStartOnce(SweetPAKEERROR) is thrown when someone sends the wrong confirmation code
\end{itemize}

The two main classes represent both parties of a connection. The first class
named SweetPAKE\_Client represents the client side and the second class name
SweetPAKE\_Server represents the server side of the connection.

\begin{itemize}
	\item SweetPAKE\_Client
	\item SweetPAKE\_Server
\end{itemize}

group.py: The module consists of a class named IntegerGroup and a private class
named \_Element. The IntegerGroup class represents a cyclic group and the
\_Element class represents an element of the group. The module of a third class
named \_Params whose purpose is to create a know big secure cyclic group.

\begin{itemize}
	\item IntegerGroup
	\item \_Element
	\item \_Params
\end{itemize}

\subsubsection{data structures}
Now, the different data structures are presented.

Beginning with the IntegerGroup class of the group.py module:
\begin{itemize}
	\item q being the order of the subgroup generated by the generators stored as a whole number
	\item p being the order of the group stored as a whole number
	\item Zero being the identity Element of the group stored as an object of class \_Element 
	\item Base1 and Base2 are two generators of the group
		stored as an object of the class \_Element
\end{itemize}

Now presenting the \_Element class:
\begin{itemize}
	\item \_group is the group the element is in stored as an object of the
		IntegerGroup class 
        \item \_e is the value of the element can have
		multiple types but is usually stored as a whole integer number
\end{itemize}

\_Params class
\begin{itemize}
	\item group stored as an object of the IntegerGroup class
\end{itemize}

Outside of classes as global variables:
\begin{itemize}
	\item I1024, I2048, I3072 are three objects of the IntegerGroup class
	\item Params1024, Params2048, Params3072 are three object of the
		\_Params class created with the three IntegerGroup object
		showed above
\end{itemize}

Continuing with data structures of sweet\_pake module. Global variables:
\begin{itemize}
	\item ClientId and ServerId are the ids of both sides which are text
	stored as a byte object 
  	\item DefaultParams is the IntegerGroup used from
		both sides are stored as objects of the class \_Params
\end{itemize}

The data structure which both classes SweetPAKE\_Client and SweetPAKE\_Server have in common:
\begin{itemize}
	\item password is the shared key stored as a byte object 
	\item ida and idb being the ids of the sides which are communicating
		with each other stored as byte object
	\item params is the integer group being used as the object of the \_Params class
	\item entropy\_f the entropy used and stored as byte object
\end{itemize}

The class SweetPAKE\_Server has one additional data\_structure:
\begin{itemize} 
 	\item database which is an associative array with all passwords as
			values and the usernames as values
\end{itemize}

\subsection{Implementation}

The section will present the most important parts of the implementation.
Beginning with the gen.py module from the honeyword generation module. 

Line 5-42 is a honeyword generation method proposed by the authors of
Honeywords \cite{juels2013honeywords}. The chaffing-by-digits method 
takes the password and creates a new password by replacing the digits with other
random digits. The function takes the password p and the number of sugarwords k
as parameters. First, an array named positions, an index i and an array
of sugarwords are created. Then, it traverses every character of the string p, if a
traversed character is a digit, its' position is appended to the array
positions. Then the next for loop is traversed k times, at every traversal the
password p is added and then every digit is replaced by a random digit. The complexity
of this method is quadratic. \(O(n^2)\).

\begin{lstlisting}[language=Python]
def gen_chaff_digits(p, k):
    positions = []
    i = 0

    sugarwords = []

    for c in p:
        if c.isdigit():
            positions.append(i)
        i += 1

    sys_random = random.SystemRandom()
    for n in range(k):
        sugarwords.append(p)
        for x in positions:
            rand = sys_random.randint(0, 9)
            sugarwords[n] = sugarwords[n][:x] 
	    	   + str(rand) 
		   + sugarwords[n][x+1:]

    sugarwords.append(p)
    return sugarwords
\end{lstlisting}

Lines 27-48 implement the chaffing-by-model method explained in the honeywords
section. It uses a large database of the most common pattern in passwords
consisting of letters. The algorithm identifies the sequences of characters in the
password that only consists of letters and then replaces them with a random word
from the large database. To retrieve a random word from the large database it uses
the get\_rnd\_pw\_word() function. The algorithm below has a linear complexity
\(O(n)\).

\begin{lstlisting}[language=Python]
def gen_by_model(p, k):
    sugarwords = []

    word_index = 0
    words = []
    for i in range(len(p)):
        if p[i].isdigit():
            if i != 0 and not p[i-1].isdigit():
                word_index += 1
        else:
            if i == 0 or p[i-1].isdigit():
                words.append(p[i])
            else:
                words[word_index] += (p[i])

    for n in range(k):
        x = p
        for word in words:
            x = x.replace(word, 
	    	get_rnd_pw_word())
        sugarwords.append(x)

    return sugarwords
\end{lstlisting}

Lines 51-59, return a random word from the large database which is stored in c\_pws.

\begin{lstlisting}[language=Python]
def get_rnd_pw_word():
    words = []
    rnd = 0
    sys_random = random.SystemRandom()
    with open("c_pws") as f:
        words = f.readlines()
        rnd = sys_random.randint(0, len(words)-1)

    return words[rnd][0:-1]
\end{lstlisting}

At lines 84 to 87, the function is given an array of sugarwords and it encrypts every element
using a scrypt hash from the library hashlib. The sugarwords in the array are replaced with
the encrypted words. The complexity of this function is linear \(O(n)\).

\begin{lstlisting}[language=Python]
def encrypt_pws(arr):
    for i in range(len(arr)):
        dk = hashlib.scrypt(bytes(arr[i], 'utf-8'), salt=b"NaCl", n=16384, r=8, p=16)
        arr[i] = dk.hex()
\end{lstlisting}

In lines 90-93, the Fisher-Yates algorithm is implemented. It takes an array and shuffles it, such
that each sugarword has a random position. The algorithm is useable for cryptographic usage and
has linear complexity \(O(n)\).

\begin{lstlisting}[language=Python]
def fisher_yates(arr):
    for i in range(1, len(arr)):
        j = random.randint(0, i)
        arr[j], arr[-i] = arr[-i], arr[j]
\end{lstlisting}

The function at lines 62-81 takes a username and a password as input. The function generates
the first 3 honeywords using chaffing-by-model and then for every generated honeyword it uses
chaffing-by-digit. The resulting array of sugarwords is then shuffled using the Fisher-Yates
algorithm and then encrypted. The username together with the sugarwords are then written
in the password file. The complexity is quadratic \(O(n)\).

\begin{lstlisting}[language=Python]
def append_user_to_file(name, pw):
    k = 3
    sugarwords = gen_by_model(pw, k)
    for i in range(len(sugarwords)):
        sugarwords += 
	   gen_chaff_digits(sugarwords[i], 
	   			k)

    fisher_yates(sugarwords)
    pw_index = 
        random.randint(0, len(sugarwords) 
		- 1)
    sugarwords[pw_index] = pw
    encrypt_pws(sugarwords)

    with open("pw_file", 'a') as file:
        file.write(name + ":")

        for words in sugarwords:
            file.write(word)
            if not word == sugarwords[-1]:
                file.write(":")

        file.write("\n")
\end{lstlisting}

Now, the sweet\_pake package is presented, by first showing some functions of
the group.py module.

As you can see in the constructor at lines 104-118. The IntegerGroup has two generators which 
is necessary for the BeePAKE protocol. In lines 117 and 118, it is double-checked that the
two generators are elements of the group.

\begin{lstlisting}[language=Python]
    def __init__(self, p, q, g1, g2):
        self.q = q  
        self.p = p 
        self.Zero = _Element(self, 1)
        self.Base1 = _Element(self, g1)
        self.Base2 = _Element(self, g2)

        self.exponent_size_bytes = 
		size_bytes(self.q)
        self.element_size_bits = 
		size_bits(self.p)
        self.element_size_bytes = 
		size_bytes(self.p)

        assert pow(g1, self.q, self.p)
		== 1
        assert pow(g2, self.q, self.p) 
		== 1
\end{lstlisting}

At lines 204-219 is the operation of the group implemented being a
multiplication. It always makes sure that no other type is calculated except an
element object type, if it does not raise an error in such cases, this would be
a vulnerability that could be exploited.

\begin{lstlisting}[language=Python]
    def _exp(self, e1, i):
        if not isinstance(e1, _Element):
            raise 
	    TypeError("E^i requires E be an element")
        assert e1._group is self
        if not isinstance(i, integer_types):
            raise 
	    TypeError("E^i requires i be an integer")
        return _Element(self, pow(e1._e, i % self.q, self.p))

    def _elementmult(self, e1, e2):
        if not isinstance(e1, _Element):
            raise 
	    TypeError("E1*E2 requires E1 be an element")
        assert e1._group is self
        if not isinstance(e2, _Element):
            raise 
	    TypeError("E1*E2 requires E2 be an element")
        assert e2._group is self
        return _Element(self, (e1._e * e2._e) % self.p)
\end{lstlisting}

The function at lines 172-174 does a xoring between two byte sequences.

\begin{lstlisting}[language=Python]
def xor(self, bytes1, bytes2):
        xor = [a ^ b for a, b in zip(bytes1, bytes2)]
        return bytes(xor)
\end{lstlisting}

The function at lines 38-50 takes as several parameters and hashes to compute two
exponent which will be used for the protocol.

\begin{lstlisting}[language=Python]
def key_to_2exponents(R, y1, y2, k, exponent_size_bytes, q):
    ikm = R.to_bytes() + y1.to_bytes() + y2.to_bytes() + k
    h = Hkdf(salt=b"", input_key_material=ikm, hash=hashlib.sha256)
    info = b"SweetPAKE FO random exponents"
    seed_rs = h.expand(info, 2 * exponent_size_bytes + 32)

    (r1_bytes, r2_bytes) = splitInHalf(seed_rs)
    r1_number = bytes_to_number(r1_bytes)
    r2_number = bytes_to_number(r2_bytes)
    r1_exponent = r1_number % q
    r2_exponent = r2_number % q
    
    return (r1_exponent, r2_exponent)
\end{lstlisting}

The function at lines 65-75 returns a vector of random session keys for each
entry in the password file.

\begin{lstlisting}[language=Python]
def secrets_to_array_of_keys(k, y1, Y2, n, length_bytes):
    ikm = y1.to_bytes() + Y2.to_bytes() + k
    h = Hkdf(salt=b"", input_key_material=ikm, hash=hashlib.sha256)
    info = b"SweetPAKE vector of keys"
    length_seed_rs = length_bytes
    seed_rs = h.expand(info, n*length_seed_rs)

    length_key = len(seed_rs) // n
    keys = []
    for index in range(n):
        keys.append(seed_rs[index*n:index*n+length_key])

    return keys
\end{lstlisting}

Now continuing with the sweet\_pake module. Looking at the client side at lines 78-93,
the function generates the secret key and the public key and stores the secret
key. After that, it creates a variable named outbound\_message which includes
the id of the side, the size of the username, the username and the public key.

\begin{lstlisting}[language=Python]
    def gen(self):
        #gen function
        group = self.params.group
        self.random_exponent = group.random_exponent(self.entropy_f)
        self.y1_elem = group.Base1.exp(self.random_exponent)
        self.y2_elem = group.Base2.exp(self.random_exponent)
        Y2_elem = self.y2_elem.elementmult(group.password_to_hash(self.pw))

        #self.outbound_message = (self.y1+self.Y2) <-- apk
        y1_bytes = self.y1_elem.to_bytes()
        Y2_bytes = Y2_elem.to_bytes()
        self.outbound_message =  y1_bytes + Y2_bytes

        username_size = len(self.username).to_bytes()

        outbound_id_and_message = self.side + username_size + self.username + self.outbound_message

        return outbound_id_and_message
\end{lstlisting}

The server side reacts to the outbound message with the function at lines 198-232. The
function parses first the message extracts the username and the apk. He then generates 
a vector of session keys using the secrets\_to\_array function. Then encrypts every
sugarword using the encryption method used in PAPKE FO \cite{bradley2019password} explained
in the PAPKE section. The implementation of the method is at lines 234-252 it produces several
secrets which are then included in an outbound message variable together with the servers' id and
the byte size of one secret.

\begin{lstlisting}[language=Python]
def enc(self, inbound_message):
        #parse inbound_messahe
        self.inbound_message = self._extract_message(inbound_message)

        #get username from message
        username_size = int.from_bytes(self.inbound_message[:1])
        self.working_with = self.inbound_message[1:username_size+1].decode('utf-8')

        self.inbound_message = self.inbound_message[username_size+1:]

        apk = self.parse_apk(self.inbound_message)
        group = self.params.group
        y1_elem = group.bytes_to_element(apk[0])
        Y2_elem = group.bytes_to_element(apk[1])
        client_pw_array = self.database[self.working_with]

        #PRF - get array of keys
        k = os.urandom(32)
        self.arr_K = group.secrets_to_array(k, y1_elem, Y2_elem, len(client_pw_array), 32)
        #self.arr_K = []
        self.ciphers = []

        #enc_function
        for i in range(len(client_pw_array)):
            gen_ciphers = self._papke_enc(group, y1_elem, Y2_elem, client_pw_array[i], self.arr_K[i])
            self.ciphers.append(gen_ciphers)
        
        #shuffle cipher array
        rp_ciphers, self.pmap = fisher_yates(self.ciphers)

        #message
        #self.outbound_message = c = (c1, c2, c3)
        self.outbound_message = b"".join(rp_ciphers)
        outbound_sid_and_message = self.side + len(rp_ciphers).to_bytes() + self.outbound_message
        return outbound_sid_and_message

def _papke_enc(self, group, y1_elem, Y2_elem, pw, session_key):
        #session_key = os.urandom(32)
        #self.arr_K.append(session_key)

        pw_to_hash = group.password_to_hash(bytes.fromhex(pw))
        y2_elem = Y2_elem.elementmult((pw_to_hash.exp(-1)))

        random_exponent = group.random_exponent(self.entropy_f)
        R_elem = group.Base1.exp(random_exponent)

        (r1, r2) = group.secrets_to_hash(R_elem, y1_elem, y2_elem, session_key)

        c1 = group.Base1.exp(r1).elementmult(group.Base2.exp(r2))
        c2 = y1_elem.exp(r1).elementmult(y2_elem.exp(r2)).elementmult(R_elem)
        c3 = group.xor(hashlib.sha256(R_elem.to_bytes()).digest(), session_key)

        C = c1.to_bytes() + c2.to_bytes() + c3

        return C
\end{lstlisting}

At lines 97-126 is the decrypt function it first parses the message and then tries to
decrypt every secret using the decrypt function of the PAPKE protocol at lines 128-148. 
If the is decrypted successfully it stores the decrypted key and returns a message including
the id and the index of the successfully decrypted key.

\begin{lstlisting}[language=Python]
    def dec(self, inbound_message):
        #parse message
        group = self.params.group
        self.inbound_message = self._extract_message(inbound_message)

        len_ciphers = int.from_bytes(self.inbound_message[:1])
        if len_ciphers < 0:
            raise ValueError("Invalid size")

        self.inbound_message = self.inbound_message[1:]
        ciphers = self._parse_array(self.inbound_message, len_ciphers)
        index = -1

        #dec
        for i in range(len(ciphers)):
            session_key_computed = self._papke_dec(group, ciphers[i])
            
            #checks if decryption is successful
            if session_key_computed != -1:
                index = i
                break

        if index == -1:
            raise ValueError("Could not decrypt")

        self.session_key = session_key_computed

        self.second_outbound_message = i.to_bytes()

        return self.side + self.second_outbound_message

    def _papke_dec(self, group, cipher_tuple):
        (c1, c2, c3) = self._parse_key(cipher_tuple)
        c1 = group.bytes_to_element(c1)
        c2 = group.bytes_to_element(c2)

        #computation
        R_elem = c2.elementmult(c1.exp(-self.random_exponent))
        session_key_computed = group.xor(c3, hashlib.sha256(R_elem.to_bytes()).digest())
        (r1, r2) = group.secrets_to_hash(R_elem, self.y1_elem, self.y2_elem, session_key_computed)

        if c1.to_bytes() != group.Base1.exp(r1).elementmult(group.Base2.exp(r2)).to_bytes():
            return -1

        return session_key_computed
\end{lstlisting}

After that, the server retrieves the key using the function at lines 258-264.

\begin{lstlisting}[language=Python]
    def retrieve_key_ask_honeychecker(self, inbound_message):
        self.second_inbound_message = self._extract_message(inbound_message)
        index = int.from_bytes(self.second_inbound_message)
        original_index = self.pmap[index]
        self.session_key = self.arr_K[original_index]

        # todo verify honeychecker

        return self.session_key
\end{lstlisting}

\subsection{Improvements}
The implementation is 90\% finished but it still can be improved. 
List of improvements:
\begin{itemize}
	\item Integration of honeychecker
	\item Improved benchmark system
	\item Unit tests
	\item Improved comments
	\item Refractor according to one style guide
	\item Include MAC authentication
	\item Improve Error Handling
\end{itemize}
\end{document}
