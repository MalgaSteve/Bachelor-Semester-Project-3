\documentclass[../main.tex]{subfiles}

\begin{document}

Since the invention of personal computers, a username and password combination
has been one of the most used authentication methods until today. Even though
it is known that it has various drawbacks and weaknesses; including, humans
being bad at making up strong and unpredictable passwords which makes it
vulnerable to brute force and dictionary attacks. A problem a system
administrator also has to solve is how to store the usernames and passwords.
This is usually done in a password file (Example Linux: /etc/shadow
/etc/passwd). In the past enterprises struggled with data breaches being open
without them realizing or realizing too late, which resulted in leakage of their
password files. This can be devastating to them and their clients. To make a
system more secure one can incorporate fake accounts in the password file,
therefore if somebody tries to enter the system using a fake account, the
system administrator will be alarmed and know that the password file is in
possession of outsiders. This report will analyze several security mechanisms
one of them being fake passwords also called honeywords proposed by Juels and
Rivest \cite{juels2013honeywords}.  

Although weak passwords are not secure they have some benefits, one of them is
being very user-friendly and easy to remember. Nowadays, several protocols make
it possible to securely authenticate peers by agreeing on a session key using a
weak password. Such protocols are called PAKE protocols
\cite{bellovin1992encrypted} \cite{bellare2000authenticated}
\cite{boyko2000provably} \cite{canetti2005universally}. This report is going to
analyze the PAPKE protocol of Bradley, Gamenisch, Jarecki, Lehmann, Neven, and
Xu \cite{bradley2019password} in detail, which is a PAKE protocol using a
public key. This protocol enables the ability to authenticate for example a
browser without needing to trust a third party by using certificates, which is
vulnerable to phishing attacks where attackers can pretend to be the
authentication server.

The paper SweetPAKE of Arriaga, Ryan and Skrobot \cite{marjan2023} shows
several approaches to combine both principles Honeywords and PAKE. They
highlight the secure approaches and implementation. This report revolves mainly
around the SweetPAKE paper, it analyzes the different approaches and offers an
implementation of one approach. The implementation serves several purposes
being to provide an example how one can implement such a protocol, the code can
also be used as template and it can also be used to benchmark the protocol, to
see if it is fast enough to operate in the real world. The implementation uses
parts of code of the SPAKE2 implementation of warner \cite{Warner2016}.

Analyzing Honeywords, PAKE protocols, and SweetPAKE, this report tries to
answer the question: How to detect if a password file is in possession of
intruders and at the same time prevent phishing attacks? 

\end{document}
