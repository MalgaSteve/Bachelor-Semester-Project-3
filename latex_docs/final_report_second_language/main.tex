\documentclass[conference, compsoc]{IEEEtran}
\usepackage{datetime}
\usepackage[style=ieee]{biblatex}
\usepackage{subfiles}
\usepackage{tabularx}
\usepackage{array}
\usepackage{amsfonts}
\usepackage{amsmath}
\usepackage{listings}
\usepackage{xcolor}
\usepackage{tikz}
\usepackage{algpseudocode}
\usetikzlibrary{shapes.geometric, arrows}

\definecolor{codegreen}{rgb}{0,0.6,0}
\definecolor{codegray}{rgb}{0.5,0.5,0.5}
\definecolor{codepurple}{rgb}{0.58,0,0.82}
\definecolor{backcolour}{rgb}{0.95,0.95,0.92}

\lstdefinestyle{mystyle}{
    backgroundcolor=\color{backcolour},   
    commentstyle=\color{codegreen},
    keywordstyle=\color{magenta},
    numberstyle=\tiny\color{codegray},
    stringstyle=\color{codepurple},
    basicstyle=\ttfamily\footnotesize,
    breakatwhitespace=false,         
    breaklines=true,                 
    captionpos=b,                    
    keepspaces=true,                 
    numbers=left,                    
    numbersep=5pt,                  
    showspaces=false,                
    showstringspaces=false,
    showtabs=false,                  
    tabsize=2
}

\lstset{style=mystyle}
\addbibresource{biblio.bib}

\begin{document}

\title{PAKE Protokolle und Täuschungs Kennwörter\\
{\small \today~-~\currenttime}}

\author{
    \IEEEauthorblockA{Steve Meireles Lopes\\
    University of Luxembourg\\
    Email: steve.meireles.001@student.uni.lu}\\
    This report has been produced under the supervision of:\\
    \IEEEauthorblockA{Marjan Skrobot\\
    University of Luxembourg\\
    Email: marjan.skrobot@uni.lu}}

\maketitle

\begin{abstract}
Das Bericht antwortet auf die Frage: Wie erkennt man, ob eine Kennwort-Datei im
	Besitz von Eindringlingen ist und kann man gleichzeitig
	Phishing-Angriffe verhindern?  Dies wird beantwortet mit SweetPAKE, das
	die Stärken von PAKE Protokolle und Honeywords kombiniert. Es stellt
	auch eine Implementierung eines SweetPAKE vor.
\end{abstract}


\section{Einleitung}
Seit der Erfindung von Rechner, wird die Benutzername und Passwort
Authentifizierungs Methode am häufigsten benutzt. Mittlerweile, weiss man dass
Menschen sehr schlecht im erfinden und merken von komplexen Passwörten die man
nicht vorhersagen kann. Es gibt etliche Methode diese Schwachstelle auszunutzen
wie zum Beispiel die "Offline Dictionary" Attacke oder einfach das tausendfache
probieren eines Passworts. In den letzen Jahre gab es auch Veröffentlichung von
Millionen von Passwörter, weil Kennwort-Datei von Firmen an falschen Hände
geraten. Ein weiteres Authentifikations Problem im Netz is das abhängige
Vertrauen an Drittanbietern die Zertifikate austeilen. 

Dieses Bericht widmet sich der Frage: Wie erkennt man, ob eine Kennwort-Datei
im Besitz von Eindringlingen ist und kann man gleichzeitig Phishing-Angriffe
verhindern? 

\section{Honeywords}
Honeywords ist eine Methode das von Rivest und Juels empfohlen
wurde\cite{juels2013honeywords}. Die Methode besteht darin mehrere "falsche
"Passwörter zu haben, die sich als richtiges Passwort tarnen, die nennt man
Honeywords. Diese Honeywords werden dann zusammen mit dem richtigen Passwort in
der Kennwort-Datei gespeichert. Die Position wird dann einer seperaten System
names Honeychecker gespeichert. Der Honeychecker checkt bei jedem Anmelde
Versuch ob der Benutzer ein getarnten Passwort eingeben, wenn ja nimmt
dementsprechende Massnahmen wie zum Beispiel dem Administrator zu alamieren. 
Die Autoren von Honeywords empfehlen auch mehrere Algorithmen um die Honeywords
zu generieren, sie achten drauf, dass die Honeywords nicht zu differenzieren ist
mit dem richtigen Passwort sodass ein ausenstehender nicht wissen kann was das
richtige Passwort ist.

\section{PAKE Protokolle}
PAKE Protokolle sind Protokolle die es ermöglichen eine sichere Kommunikation
herzustellen,obwohl man schwache Passwörter benutzt. Solche Protokolle benutzen
zyklische Gruppen wie auch das Diffie-Hellman Prinzip.

Die Autoren von PAPKE \cite{bradley2019password}, und Bradley, Tatiana und
Camenisch, Jan und Jarecki, Stanislaw und Lehmann, Anja und Neven, Gregory und
Xu, Jiayu, empfehlen zum Beispiel solch einen Protokoll. Jedoch fügen sie noch
eine Primitive hinzu und zwar fokussieren sich mit dem generieren von Lang-Zeit
Schlüsseln die eine sichere End-zu-End Verschlüsselung garantieren. Sie
empfehlen zwei Schemas die das PAPKE Protokoll befolge, das PAPKE-IC und das
PAPKE-FO.

\section{SweetPAKE}
SweetPAKE empfohlen von Arriage, Ryan und Skrobot, kombiniert die Stärken
von Honeywords und PAKE Protokolle. Die Autoren von SweetPAKE empfehlen
eine Naive Variante und eine sichere Variante. Die sichere Variante namens 
BeePAKE benutzt als Basis, das PAPKE protokoll.

Schritte:

Lass Alice und Bob zwei Parteien sein die ein Schlüssel teilen wollen.

1. Alice generiert ein öffentlicher Schlüssel und ein privater mit Hilfe von
ihrem Passwort. Sie schickt das öffentliche Schlüssel zu Bob.

2. Bob erstellt dann ein Schlüssel für jedes Kennwort in der Kennwort-Datei
mit Hilfe vom öffentlichen Schlüsseln. Dieses Liste von Schlüsseln wird dann an
Alice geschickt.

3. Alice entschlüsselt alle Inhalte von der List bis sie ein gültigen Schlüssel
bekommt und speichert die Position. Sie schickt dann die Position weiter an Bob.

4. Bob schickt nun die Position an den Honeychecker, der checkt ob ein Honeyword oder
das richtege Passwort bentutz wurde.

\section{Technische Arbeit}
Diesen Abschnitt beeinhaltet den technischen Teil des Projektes. Es besteht aus der 
Implentierung des BeePAKE Protokolls \cite{marjan 2023}.

Die Anwendung sollte eine Implementierung von BeePAKE \cite{marjan2023} sein,
die in dem vorherigen Abschnitt diskutiert und analysiert wurde. Die
Implementierung sollte auf jedem Betriebssystem funktionieren. Es sollte mit
der Python Version 2 oder höher laufen. Das BeePAKE-Protokoll wird mit dem
PAPKE-FO \cite{bradley2019password} Protokoll implementiert, das in vorherigen
Abschnitten erläutert wurde. Das Projekt wird eine Beispieldatei enthalten, die
zeigt, wie die Implementierung verwendet und getestet werden kann. Es wird zwei
Parteien haben und zeigen, wie beide einen Schlüssel teilen, indem sie den
bereitgestellten Code verwenden.

Das Protokoll wird in vier Teile unterteilt, von denen jeder einen Schritt des
Protokolls darstellt. Jeder Teil hat seine eigene Funktion:

\begin{itemize}
	\item generate(): Erster Schritt des BeePAKE-Protokolls
	\item encryption(): Verschlüsselungsschritt
	\item decryption(): Entschlüsselungsschritt
	\item retrieve\_key(): Schlüsselabrufschritt
\end{itemize}

Das Protokoll benötigt auch drei Funktionen des Verschlüsselungsschemas des PAPKE-Protokolls:
\begin{itemize}
	\item papke\_generate()
	\item papke\_encryption()
	\item papke\_decryption()
\end{itemize}


Die Funktion generate() sollte die folgende Vorlage verwenden:
\begin{itemize}

\item Beschreibung: Erster Schritt des BeePAKE-Protokolls
\item Parameter: Keine Parameter
\item Vorbedingung: Das Protokoll wurde gestartet
\item Nachbedingung: Generiert einen geheimen Schlüssel und einen öffentlichen
	Schlüssel und gibt die ausgehende Nachricht zurück, die gesendet wird
\item Auslöser: Eine Partei fordert den Schlüsselaustausch mit einem Server an
\end{itemize}

Die folgende gezeigte Funktion gen() ist der erst Schritt des BeePAKE Protokoll
es generiert eine öffentlichen Schlüssel und eine privaten und speichert dies.

\begin{lstlisting}[language=Python]
def gen(self):
        #gen function
        group = self.params.group
        self.rundom_exponent = group.rundom_exponent(self.entropy_f)
        self.y1_elem = group.Base1.exp(self.rundom_exponent)
        self.y2_elem = group.Base2.exp(self.rundom_exponent)
        Y2_elem = self.y2_elem.elementmult(group.password_to_hash(self.pw))

        #self.outbound_message = (self.y1+self.Y2) <-- apk
        y1_bytes = self.y1_elem.to_bytes()
        Y2_bytes = Y2_elem.to_bytes()
        self.outbound_message =  y1_bytes + Y2_bytes

        username_size = len(self.username).to_bytes()

        outbound_id_und_message = self.side + username_size + self.username + self.outbound_message

        return outbound_id_und_message
\end{lstlisting}

Die Funktion enc() entschlüsselt jedes Passwort in der Datei mit Hilfe von der
Nachrichter von der zweiten Partei und sendet die nötigen Information zu
underen Partei.

\begin{lstlisting}[language=Python]
def enc(self, inbound_message):
        #parse inbound_messahe
        self.inbound_message = self._extract_message(inbound_message)

        #get username from message
        username_size = int.from_bytes(self.inbound_message[:1])
        self.working_with = self.inbound_message[1:username_size+1].decode('utf-8')

        self.inbound_message = self.inbound_message[username_size+1:]

        apk = self.parse_apk(self.inbound_message)
        group = self.params.group
        y1_elem = group.bytes_to_element(apk[0])
        Y2_elem = group.bytes_to_element(apk[1])
        client_pw_array = self.database[self.working_with]

        #PRF - get array of keys
        k = os.urundom(32)
        self.arr_K = group.secrets_to_array(k, y1_elem, Y2_elem, len(client_pw_array), 32)
        #self.arr_K = []
        self.ciphers = []

        #enc_function
        for i in range(len(client_pw_array)):
            gen_ciphers = self._papke_enc(group, y1_elem, Y2_elem, client_pw_array[i], self.arr_K[i])
            self.ciphers.append(gen_ciphers)
        
        #shuffle cipher array
        rp_ciphers, self.pmap = fisher_yates(self.ciphers)

        #message
        #self.outbound_message = c = (c1, c2, c3)
        self.outbound_message = b"".join(rp_ciphers)
        outbound_sid_und_message = self.side + len(rp_ciphers).to_bytes() + self.outbound_message
        return outbound_sid_und_message
\end{lstlisting}

Dec() nimmt eine Nachricht an und entschlüsselt die Liste von Schlüsseln bis sie eine gültige
raus bekommt. Schlussendlich sendet sie die Position zur underen Partei.

\begin{lstlisting}[language=Python]
def dec(self, inbound_message):
        #parse message
        group = self.params.group
        self.inbound_message = self._extract_message(inbound_message)

        len_ciphers = int.from_bytes(self.inbound_message[:1])
        if len_ciphers < 0:
            raise ValueError("Invalid size")

        self.inbound_message = self.inbound_message[1:]
        ciphers = self._parse_array(self.inbound_message, len_ciphers)
        index = -1

        #dec
        for i in range(len(ciphers)):
            session_key_computed = self._papke_dec(group, ciphers[i])
            
            #checks if decryption is successful
            if session_key_computed != -1:
                index = i
                break

        if index == -1:
            raise ValueError("Could not decrypt")

        self.session_key = session_key_computed

        self.second_outbound_message = i.to_bytes()

        return self.side + self.second_outbound_message

\end{lstlisting}

Die nächst gezeigte Funktion entnimmt aus der Nachricht und nimmt die nötigen Schritten
um das Schlüssel zu bekommen.

\begin{lstlisting}[language=Python]
def retrieve_key_ask_honeychecker(self, inbound_message):
        self.second_inbound_message = self._extract_message(inbound_message)
        index = int.from_bytes(self.second_inbound_message)
        original_index = self.pmap[index]
        self.session_key = self.arr_K[original_index]

        # todo verify honeychecker

        return self.session_key
\end{lstlisting}

\section{Schlussfolgerung}
Als Schlussfolgerung kann man sage, dass SweetPake eine gute Antwort auf die
Frage; Dieses Bericht widmet sich der Frage: Wie erkennt man, ob eine
Kennwort-Datei im Besitz von Eindringlingen ist und kann man gleichzeitig
Phishing-Angriffe verhindern? ist.

Mit Hilfe der Implementierung kann man dies auch austesten und möglicherweise
Benchmarks ausführen.

\printbibliography

\end{document}
