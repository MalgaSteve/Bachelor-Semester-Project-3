\documentclass[conference,compsoc]{IEEEtran}
\usepackage{datetime}
\usepackage{caption}
\usepackage{listings}
%\usepackage{xcolor}

\usepackage[backend=biber]{biblatex}
\addbibresource{sample.bib}

\hyphenation{op-tical net-works semi-conduc-tor}

\begin{document}

\title{Enhanced Password Security: PAKE and Anti-Phishing\\
{\small \today~-~\currenttime}}
 
\author{\IEEEauthorblockN{Meireles Lopes Steve}
\IEEEauthorblockA{University of Luxembourg\\
Email: steve.meireles.001@student.uni.lu}
\\
{\bf This report has been produced under the supervision of:}\\
\IEEEauthorblockN{Skrobot Marjan}
\IEEEauthorblockA{University of Luxembourg\\
Email: marjan.skrobot@uni.lu}%
}

\maketitle

\begin{abstract}

In today's world, password security is very important. Most people have weak
	passwords which exposes them to brute force attacks. Furthermore
	malicious attackers also often use phishing attacks. These attacks
	expose a huge vulnerability and therefore a risk for the data of a lot
	of people. This project is to answer the scientific question: In the
	context of client-to-server authentication over an insecure network,
	how does one detect if a password file on the server side (containing
	usernames and passwords) was compromised by intruders and at the same
	time prevent phishing attacks on clients?

\end{abstract}

\IEEEpeerreviewmaketitle

\section{Main required competencies} 

\subsection{Scientific main required competencies} 
For the scientific deliverable, it is essential for the student to possess a
solid foundation in cryptography and security. This includes a thorough
understanding of key concepts such as password security, symmetric and
asymmetric encryption, and key exchange protocols. The student should be
well-versed in the mechanisms that underpin these concepts, enabling them to
analyze, design, and implement secure authentication systems.

In addition to cryptographic knowledge, a grasp of mathematical concepts is
imperative. Competency in areas such as set theory and mathematical notations
is essential to navigate the mathematical underpinnings of encryption
algorithms and security protocols. These mathematical skills are integral to
the development and analysis of secure authentication mechanisms.

\subsection{Technical main required competencies} 
The technical deliverable, while sharing the core competencies with the
scientific counterpart, demands an additional skill set. In addition to
cryptography and security expertise, the student tasked with the technical
deliverable must be proficient in the programming language Python.

Python is the chosen medium through which the theoretical concepts will be
translated into practical, functional solutions. The student should have a
strong command of Python programming, enabling them to effectively implement
authentication systems, manage password files, generate decoy passwords, and
integrate the SweetPAKE protocol.

These technical competencies encompass not only an understanding of
cryptographic principles but also the ability to translate this knowledge into
tangible, working software. Proficiency in Python is the foundation that bridges
theoretical security concepts to real-world application, making the deliverable
a practical and impactful contribution to the field of authentication security.

\section{ A Scientific Deliverable 1 } 

In the realm of cybersecurity, a significant number of enterprises rely on
username and password logins as the primary means of authenticating their users
or employees. This method extends a wide array of technologies and systems
including from email services to online banking. Everybody has a few
technologies that require them to authenticate with their username and
password. However is commonly known that this is often exploited by attackers
through weak passwords, brute forcing or even phishing attacks. It has a lot of
vulnerabilities for example weak passwords, often a result of human tendencies
to choose easily guessable combinations, present a security risk. The threat is
not limited to weak passwords but also by various tactical attacks including
brute force attacks, where automated scripts or tools systematically guess
passwords until they find the correct one. Also popular are phishing attacks,
which involve social engineering techniques, where criminals disguise
themselves as trustworthy parties to deceive individuals into revealing details
about them or their login credentials. This can take the forms of fake
websites, fake emails or misleading messages.

\subsection{The scientific question} These vulnerabilities are particularly
pronounced in the context of client-to-server authentication over insecure
networks, where the transmission of login credentials is potentially exposed to
eavesdropping and malicious interception. This brings us to the scientific
question: In the context of client-to-server authentication over an insecure
network, how does one detect if a password file on the server side (containing
usernames and passwords) was compromised by intruders and at the same time
prevent phishing attacks on clients? The deliverable embarks comprehensive
exploration of various security approaches. The project encompasses three key
areas: password security, and the prevention of brute force and phishing
attacks.

\subsection{Honeywords and Decoy Passwords} The project will study password
security, secure storage of passwords and brute forcing attacks. This will be
done by exploring the concepts of "honeywords" as introduced in the Honeyword
paper \cite{juels2013honeywords} by Rivers Juels. The paper introduces how you can add decoy
passwords to the password file and suggests methods to generate these passwords
such that even a skilled attacker can not guess the correct password from the
decoy passwords. Then it introduces the concept of a honeychecker which stores
the index of the correct passwords and communicates with the server that's
working with the client.

\subsection{PAKE} Furthermore, it looks into ways to enhance current password
authentication protocols from sending a hash of a user's password over TLS to
utilising PAKE (Password Authentication Key Exchange). This transition to PAKE
introduces a a robust layer of security that prevents phishing attacks by
ensuring a secure client-to-server communication. 

\subsection{SweetPAKE protocol} On top of all that the paper is going to
examine the SweetPAKE protocol from Arriage, Ryan and Skrobot \cite{sweetpake}
which uses PAKE and enhances it with a decoy password which adds a password
file leakage detection mechanism at the server side. The SweetPAKE uses PAPKE\cite{papke}
(Password-Authenticated Public Key Encryption) principles, these principles are
going to be analyzed and described.

\subsection{Deliverable Structure} To answer the question, the deliverable will
be split into three main sections. The first section will try to answer the
question: How to detect if a password file is compromised? This will cover in
detail how decoy passwords work and how an external server can alarm the
administrator if an attacker compromises the password file. The section also
talks about the different decoy password generation methods and their
probabilities of being guessed. 

The second section goes into detail about the technologies of PAKE. How it
works and which protocols are being used nowadays. This will be done with
graphs showing visually the process and the section will explain the
mathematical side of such protocols. PAKE provides several properties, which
are that a man-in-the-middle does not have enough information to execute a
brute force guessing attack. Which therefore provides strong security using
weak passwords.

The last section will explain how to improve the security of the methods talked
about in the first section by adding a PAKE protocol in the client-server
authentication. It goes into detail about the SweetPAKE \cite{sweetpake}
protocol and PAPKE\cite{papke}. This makes an attack a lot harder and it provides
additional protection against phishing attacks to the already strong idea
established by Riverst Juels. The section may also have a benchmark of the
implementation done in the technical part of this paper.

\subsection{Conclusion} In a landscape defined by the reliance on username and
password-based authentication and the security of these credentials is very
important. This project's exploration into honeywords, PAKE protocols, and the
SweetPAKE enhancement offers a promising avenue to strengthen password security
and prevent phishing attacks. By combining innovative strategies, we aspire to
elevate the standards of authentication and significantly enhance the security
of digital interactions.

\section{A Technical Deliverable 1 } The technical deliverable will be the
implementation of a decoy password generator according to proposed protocols
from the Honeyword paper \cite{juels2013honeywords} of Riverst Jules which is
discussed in the scientific part. Additionally, it integrates the SweetPAKE
protocol \cite{sweetpake} suggested by Arriaga, Ryan and Skrobot.

\subsection{The programming language} The entire technical part is realized
through the versatile and widely used programming language Python. Python
offers an ideal platform for these implementations due to its ease of use,
extensive libraries, and cross-platform compatibility. It ensures that the
solution remains accessible to a wide range of users and organizations.

\subsection{Motivation and Significance} This project is beneficial to the
security world considering there are not a a lot of open-source PAKE
implementations despite its security value. Combining PAKE with a decoy
password generator which further enhances password security does not exist in
the current market. It has the potential to reshape the landscape of digital
security. It not only strengthens traditional password security but also
prevents phishing attacks by malicious parties. It is a formidable asset in the
ongoing battle against malicious attacks in our interconnected world.

\subsection{Project structure} The deliverable will consist of two parts and
try to combine both. The first part is the implementation of the ideas of
Rivers Juels \cite{juels2013honeywords}. The second part is the implementation
of the SweetPAKE introduced by Arriage, Ryan and Skrobot \cite{sweetpake}. 

\subsubsection{Part 1: Honeywords Implementation} The aim of the project is to
give a practical example of how the suggested password-generation methods of
Riverst Juels can be implemented. In addition to that it will simulate the
actions of the Honeychecker (extern server).

The functions the first part has to have:

Firstly, there will be a password file and the deliverable has to be able to
add a user to it. The function add\_user(name, pw) will be responsible for
that. This function takes a name and a password as a parameter. It generates an
array of honeywords consisting of real passwords and decoy passwords. After
generating, the program will shuffle the array with the Fisher-Yates algorithm.
Then it writes the username and the passwords separated by a ":" into the
password file.

Secondly, the Honeychecker functions. The Honeychecker checks(i) if the integer
i corresponds to the index of the correct passwords. It can also receive(i) a
new index when a user changes passwords or a new user is added to the database.
This function changes or adds the new index of the user to the index file (the
file which stores the indexes of the correct password of the corresponding
users).

The function of the deliverable is coded as suggested by the honeyword paper
such that it can be said to be secure.

\subsubsection{Part 2: SweetPAKE implementation} The second part will consist
of a PAKE implementation. This will ensure security against phishing attacks.
Two parties can safely exchange long-term keys without a middleman being able
to trick one of the parties.

The functions of the second part:

Two parties have to agree on a key. The project will be able to communicate and
agree on a session key this will be done following the SweetPAKE paper
\cite{sweetpake}. This involves several steps and functions with the
mathematical background behind it.

After successfully implementing a PAKE. The SweetPAKE will be fully implemented
by combining both previous parts. This implementation will then be a good
example of an implementation of both Honeywords and SweetPAKE.

\subsection{Bonus} If the students have time. A benchmark may be added to the
SweetPAKE protocol which tests the speeds of several different encryption
algorithms and the difference between the speed of using the SweetPAKE protocol
and using the current methods.

\subsection{Existing code used} To be able to achieve this program will use the
open-source project python-spake2 of Warner as a reference. It will use the
standard library random and hashlib to encrypt and generate passwords.

\subsection{Conclusion} In conclusion, this technical deliverable merges the
power of Honeywords and the SweetPAKE protocol to create a robust and
innovative approach to authentication security. It offers a practical solution
to enhance password security, prevent phishing attacks and strengthen the
authentication process in an interconnected digital world.

\printbibliography

\clearpage

\end{document}
